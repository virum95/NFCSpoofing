\chapter{Introducción}\label{ch:introduction}
En este capítulo trataremos temas introductorios sobre la tecnología NFC, como su definición, formas de funcionar o usos. Servirá para tener una idea sobre como funciona esta tecnología y para tratar futuros temas de la seguridad y vulnerabilidades.
\clearpage
\section{NFC}
NFC significa \textit{Near Field Communication}. Se trata de una tecnología inalámbrica que deriva de las tarjetas RFID, utilizadas en sistemas de transporte o de seguridad de algún establecimiento.\par
NFC es una plataforma abierta pensada desde el inicio para el teléfono o dispositivos móviles. Tanto su tasa de velocidad (424kbit/s) como su alcance (20m) son muy bajos, ¿Por qué se utiliza esta tecnología entonces?\par
Su punto fuerte esta en la velocidad de la comunicación, que es casi instantánea y no necesita de un emparejamiento previo. Además, el uso es transparente para los usuarios y los equipos con NFC son capaces de enviar y recibir al mismo tiempo.\\\\
Las tecnologías NFC tienen dos formas de funcionar:
\begin{itemize}
	\item \textbf{Activo:} ambos equipos con chip NFC generan un campo electromagnético e intercambian datos.
	\item \textbf{Pasivo:} solo hay un dispositivo activo y el otro aprovecha ese campo para intercambiar la información.
\end{itemize}
\section{Usos}
La premisa básica a la que se acoge el uso de la tecnología NFC es aquella situación en la que es necesario un intercambio de datos de forma inalámbrica. Lo usos que más futuro tienen son la identificación, la recogida e intercambio de información y sobre todo, el pago.
\begin{itemize}
	\item \textbf{Identificación:} acceso a lugares donde es preciso identificarse podría llevarse a cabo mediante el teléfono o con una tarjeta con NFC.
	\item \textbf{Recogida/intercambio de datos:} marcar un lugar en un mapa, recibir información de un evento o establecimiento es inmediato.
	\item \textbf{Pago:} pagar con tarjetas \textit{contactless} o con el teléfono móvil convierte esta tarea en algo realmente cómodo.
\end{itemize}
Sin embargo, la comodidad es un gran enemigo de la seguridad.

\subsection{Transferir fotos, vídeo o música}
Aunque existen otras alternativas a la hora de pasar contenidos multimedia de un dispositivo móvil a otro (Bump es una de las más interesantes de los últimos tiempos), NFC es (o al menos era originalmente) una forma muy interesante de dar acceso a esta capacidad.

De hecho, los chips NFC que comienzan a integrarse en algunos portátiles como el HP Envy 14 Spectre o en cámaras de fotos como las Panasonic Lumix DMC-ZS30 y DMC-TS5 que se comercializarán a partir de este mes permiten utilizar estos chips para transferir fotos, vídeos y música con este estándar, aunque probablemente lo combinen con Bluetooth y WiFi Direct para poder aumentar las velocidades de transferencia. 

\subsection{Identificación y control del coche}
En el pasado CES se pudo ver un Porsche Carrera con un chip NFC integrado y con un pequeño ordenador basado en el sistema operativo de tiempo real QNX que disponía de una cuna para situar el smartphone con conectividad NFC. Al hacerlo el teléfono empezaba a recargarse, pero además se establecía la conexión para poder reproducir la música del móvil a través de los altavoces del coche, o realizar llamadas utilizando la agenda de contactos del smartphone.

Otra de las posibilidades consistiría en usar el móvil como una llave de acceso al coche, lo que permitiría convertir a los smartphones como sistemas redundantes (la llave convencional no se eliminaría) para poder acceder al interior del coche e incluso para poder encender el motor y desplazarnos con él. Orange y Opel ya lanzaron un sistema preliminar en este sentido que demuestra que esta es otra de las posibilidades de futuro reales de la tecnología NFC.

\subsection{Cajeros automáticos 2.0}
Otra alternativa en el ámbito de la identificación está en la capacidad de usar esta tecnología para comenzar una sesión en un cajero automático con la que poder sacar dinero. Al acercar nuestro terminal a la pantalla de un cajero con NFC, se realizaría la negociación inicial de la conexión para identificarnos y pedirnos nuestro correspondiente PIN.

Esta alternativa de nuevo se situaría como un sistema redundante que evitaría tener que utilizar la tarjeta de débito o crédito, y haría a menudo más cómoda la operación de acceder y utilizar un cajero automático.

\subsection{Compras más allá de los códigos QR}
Los códigos QR siguen teniendo validez para facilitar algunos procesos de compra, pero el hecho de tener que "escanear" los códigos para luego acceder a las opciones que nos brinda ese código QR resulta algo incómodo comparado con la capacidad que tiene la tecnología NFC de transmitir esos datos automáticamente en cuanto acercásemos nuestro terminal a una etiqueta NFC con la información de ese producto.

Así, al realizar esa transferencia de información podríamos localizar determinados artículos en una tienda, pedir la ayuda de un asistente, o tratar de aprovechar cupones y ofertas si están disponibles en el sistema al realizar el pago. 

\subsection{Identificación en eventos}
En el propio Mobile World Congress de Barcelona hemos visto como las posibilidades de identificación de la tecnología NFC son idóneas para mejorar los procesos de registro y control de acceso a todo tipo de eventos. Las llamadas NFC Badge eran acreditaciones con un chip NFC que permitían a los que las portaban poder acceder al recinto de la feria directamente y sin tener que mostrar repetidamente la acreditación física convencional.

Ese mismo sistema es el que poco a poco se va implantando --o se podría implantar al menos como opción-- en otros eventos de todo tipo, tales como eventos deportivos, conciertos, acceso a hospitales y, por supuesto, acceso a oficinas de trabajo en las que además esa capacidad se combinaría con los sistemas de control de las jornadas laborales que muchas empresas utilizan. 

\subsection{Pagos móviles}
La última posibilidad es sin duda de la que más se habla: los sistemas de pago móviles que hacen uso de la tecnología NFC llevan tiempo en desarrollo y pruebas, y de hecho hay implantaciones funcionando desde hace tiempo.

El servicio Google Wallet es probablemente el mejor ejemplo de esa ambición por proporcionar métodos de pago móviles de forma inalámbrica. En Estados Unidos la tecnología va por buen camino --200.000 comercios con sistemas de pago inalámbricos lo demuestran-- pero de momento ese desarrollo no ha sido exportado a otros países. Los sistemas alternativos de Apple (Passbook) y de Samsung (el recién presentado Wallet, aunque menos ambicioso que el servicio de Google) persiguen el mismo objetivo, aunque su aplicación práctica real aún está por demostrarse.

Sin embargo, NFC sí es la base de muchos proyectos de grandes empresas financieras tales como Visa o Mastercard, y quizás en este 2013 comencemos a ver con cierta frecuencia a usuarios que pagan el transporte público (o el taxi) con su móvil vía NFC. En España ya hay ejemplos prácticos como el de La Caixa o Banesto y su implantación del sistema de pago contactless, con tarjetas de crédito que llevan implantadas el chip NFC y que permiten realizar pagos inalámbricos, más cómodos y que demuestran que esta tecnología puede ser una alternativa válida de futuro. 

\subsection{Etiquetas NFC}
Las etiquetas (a menudo adhesivas) NFC permiten demostrar de nuevo las posibilidades de la tecnología al actuar como disparadores condicionales que permiten activar ciertos procesos en nuestro dispositivo móvil.

El escenario clásico sería el de tener una etiqueta en alguna pared (o varias) de nuestra casa para que al acercar el teléfono este habilitase la conectividad WiFi y Bluetooth, y que tuviéramos otra en la mesilla de noche que hiciera que al situar el smartphone al lado éste entrase en modo silencioso y se activase el despertador.

Un ejemplo de implementación práctica la tenemos en los Xperia SmartTags, que precisamente adaptan el perfil del teléfono según la etiqueta (a 14,90 euros cada un pack de cuatro, eso sí) a la que acerquemos el smartphone. Podríamos tener una etiqueta en el coche que activase el navegador GPS, y otras como las que hemos citado en casa para activar esos distintos perfiles. Lo mismo ocurre con las Samsung TecTiles, otro sinónimo de estas ingeniosas etiquetas inteligentes que tienen como objetivo hacernos la vida un poquito más fácil.

