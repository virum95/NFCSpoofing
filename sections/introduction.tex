\chapter{Introducción}\label{ch:introduction}
En este capítulo trataremos temas introductorios sobre la tecnología NFC, como su definición, formas de funcionar o usos. Servirá para tener una idea sobre como funciona esta tecnología y para tratar futuros temas de la seguridad y vulnerabilidades.
\clearpage
\section{NFC}
NFC significa \textit{Near Field Communication}. Se trata de una tecnología inalámbrica que deriva de las tarjetas RFID, utilizadas en sistemas de transporte o de seguridad de algún establecimiento.\par
NFC es una plataforma abierta pensada desde el inicio para el teléfono o dispositivos móviles. Tanto su tasa de velocidad (424kbit/s) como su alcance (20m) son muy bajos, ¿Por qué se utiliza esta tecnología entonces?\par
Su punto fuerte esta en la velocidad de la comunicación, que es casi instantánea y no necesita de un emparejamiento previo. Además, el uso es transparente para los usuarios y los equipos con NFC son capaces de enviar y recibir al mismo tiempo.\\\\
Las tecnologías NFC tienen dos formas de funcionar:
\begin{itemize}
	\item \textbf{Activo:} ambos equipos con chip NFC generan un campo electromagnético e intercambian datos.
	\item \textbf{Pasivo:} solo hay un dispositivo activo y el otro aprovecha ese campo para intercambiar la información.
\end{itemize}
\section{Usos}
La premisa básica a la que se acoge el uso de la tecnología NFC es aquella situación en la que es necesario un intercambio de datos de forma inalámbrica. Lo usos que más futuro tienen son la identificación, la recogida e intercambio de información y sobre todo, el pago.
\begin{itemize}
	\item \textbf{Identificación:} acceso a lugares donde es preciso identificarse podría llevarse a cabo mediante el teléfono o con una tarjeta con NFC.
	\item \textbf{Recogida/intercambio de datos:} marcar un lugar en un mapa, recibir información de un evento o establecimiento es inmediato.
	\item \textbf{Pago:} pagar con tarjetas \textit{contactless} o con el teléfono móvil convierte esta tarea en algo realmente cómodo.
\end{itemize}
Sin embargo, la comodidad es un gran enemigo de la seguridad.
